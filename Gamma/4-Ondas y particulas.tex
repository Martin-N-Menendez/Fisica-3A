\documentclass{article}
\usepackage[utf8]{inputenc}
\usepackage[spanish]{babel}
\usepackage{amsmath, amsfonts, amssymb}
\usepackage[pdftex]{color,graphicx}

\begin{document}

En 1924 de Broglie tiró la primera piedra: Si, para el electromagnetismo clásico, la radiación son ondas, pero por los efectos fotoelétricos y de Compton tenemos que son partículas, pueden, el resto de las partículas, comportarse como ondas bajo circunstancias específicas? Las características de partículas aparecen cuando la radiación interactúa con otras partículas, mientras que las características de ondas aparecen cuando se estudia la propagación de la radiación. De Broglie optó por pensar que la radiación son partículas que se mueven orquestadas por las características de alguna onda asociada. De Broglie empezó a investigar bajo la hipótesis de que las partículas tienen una \emph{pilot wave} (¿onda piloto?) asociada a la partícula que gobierna sus movimientos en forma de onda.

A pesar de la extensa cantidad de investigaciones hechas acerca del movimiento de una partícula, nunca se habían encontrado evidencias de que se moviesen como ondas. Sin embargo, considerando el caso de la difracción (donde, que se vean ondas o rayos (partículas) depende de la relación entre $\lambda$ y la apertura de la ranura) de Broglie optó por pensar que, si estos comportamientos como ondas no se habían observado en los movimientos de las partículas era porque las longitudes de onda eran demasiado pequeñas para ser apreciadas.

De Broglie postuló que, para una determinada partícula existe una \emph{pilot wave} que depende de la energía relativista de la partícula $E$ y de su momento $p$ mediante las ecuaciones $\lambda=h/p$ y $\nu=E/h$. Además, dijo que las partículas se moverían de acuerdo a las características de propagación de esta \emph{pilot wave}.

La velocidad de propagación de las \emph{pilot waves} es $w=E/p=c\sqrt{1+(m_{0}c/p)^{2}}$ según la energía relativista. A priori, parecería que una partícula no podría seguirle el ritmo a su \emph{pilot wave} debido a que esta última se mueve a una velocidad mayor a la de la luz.

Acá viene una parte medio flashera de lo que entiendo: se supone que la \emph{pilot wave} no es una sola onda, sino que son un grupo de ondas de distintas frecuencias y amplitudes de tal forma que, cuando las sumás, su velocidad en conjunto $v$ es igual a la velocidad de la partícula, de tal forma que $w=c^{2}/v$ y, además, en las zona de mayor intensidad de la suma de las ondas, debe encontrarse la partícula.

Como la onda de de Broglie viene dada por una $\lambda = h/p$ sólo podremos observar su comportamiento ondulatorio cuando el momento $p$ sea lo suficientemente pequeño como para que la longitud de onda $\lambda$ sea apreciable.

Dada la relación entre el radio del estado de energía mínima para un átomo de hidrógeno y la longitud de onda predicha por de Broglie, deberíamos poder observar que el electrón se mueve de forma ondulatoria. Sin embargo, en este caso particular, el electrón está unido al núcleo y, por lo tanto, su grupo de \emph{pilot waves} no se trasladará a otras regiones del espacio sino que formarán \emph{ondas estacionarias}. De Broglie propuso que las propiedades de una onda estacionaria serían la razón por la cual el momento cinético del electrón está cuantizado $L=nh/2\pi$, tal y como había dicho Bohr.

\[ L=pr=\frac{nh}{2\pi} \quad \wedge \quad p=\frac{h}{\lambda} \Longrightarrow 2\pi r=n \lambda \]

Por lo tanto, podemos observar que los estados de energía permitidos serán aquellos cuya circunferencia ($2\pi r$) contenga un múltiplo entero de longitudes de onda.

El requerimiento de que las \emph{pilot waves} asociadas a una partícula que describe un movimiento periódico sean ondas estacionarias es equivalente a que el movimiento de dicha partícula cumpla con las condiciones de cuantización de Wilson-Sommerfeld. Además, la independencia temporal de las ondas estacionarias asociadas a un electrón en uno de sus estados de energía permitidos, permiten entender porqué el movimiento descripto por la onda estacionaria no causa que el electrón emita radiación electromagnética.

(\ldots)

Principio de incertidumbre: $\Delta x \Delta k = 1$

Esto surge de algo así como que al sumar los distintos \emph{pilot waves} asociados a la partícula, voy a tener una región donde la partícula puede llegar a estar porque, en esa región, la intesidad es más alta.

Principio de incertidumbre: $\Delta x \Delta p_{x} \geq h$.

Principio de incertidumbre: $ \Delta t \Delta E \geq h$

\textbf{Nota Bene.} La dependencia de incerteza entre la posición inicial y final (luego de cierto tiempo $t$) es inversamente proporcional. Con lo cual, si logro medir con gran precisión la posición inicial de una partícula, luego no tendré gran precisión al intentar medir la posición final.

\end{document}