\documentclass{article}
\usepackage[utf8]{inputenc}
\usepackage[spanish]{babel}
\usepackage{amsmath, amsfonts, amssymb}
\usepackage[pdftex]{color,graphicx}

\begin{document}

\textbf{Situación clásica:}
Newton, Maxwell, Transformación de Galileo. Había un pequeño problema con la velocidad de la luz, que daba igual en cualquier sistema y no se condecía con la transformación de Galileo.

\textbf{Einstein:}
Suponiendo esto, agarró Einstein, generalizó la inercialidad de los sist. de referencias y empezó a modificar la transformación de galileo.
Primero, la idea de que el tiempo es el mismo en cualquier sistema de referencia. De esta forma definió la simultaneidad en base a la velocidad de la luz. (pág. 17)

\textbf{Soluciones de Einstein:}
\begin{itemize}
\item Las direcciones perpendiculares al movimiento de los sistemas de referencia, se conservan.
\item Los tiempos de eventos que ocurren en el mismo lugar tienen una diferencia de $1/\sqrt{1-v^{2}/c^{2}}$ respecto de cada sistema de referencia. El \emph{proper time} es el medido por el sistema de referencia que observa al evento ocurrir en el mismo sitio.
\item Las direcciones medidas paralelas al movimiento de los sistemas de referencia difieren en una cantidad $\sqrt{1-v^{2}/c^{2}}$. La longitud medida desde el sistema de referencia que está quieto es la \emph{proper length}.
\end{itemize}

\[ x = vt + x' \sqrt{1-v^{2}/c^{2}} \]
\[ x' = \frac{x-vt}{\sqrt{1-v^{2}/c^{2}}} \]

\[ t =  \frac{t'+x' v/c^{2}}{\sqrt{1-v^{2}/c^{2}}} \]
\[ t' = \frac{t-xv/c^{2}}{\sqrt{1-v^{2}/v^{2}}} \]

\[ y' =y \] 

\[ z' =z \]

Dado todo este quilombo, resulta que hay que suponer que la masa es una función de la velocidad sino las ecuaciones de Newton NO cierran, y las de Maxwell tienen que cerrar sí o sí porque Einstein dijo que la velocidad de la luz no cambiaba.

Después de delirarla un rato largo, llega a que la masa es:

\[ m(v) = m_{0}/\sqrt{1-v^{2}/c^{2}} \]

La energía relativista de una partícula es

\[ E(v) = E_{c} (v) + E (0) \]

siendo $E(0)$ la energía intrínseca de la partícula y $E_{c}(v)$ la energía cinética.
\[ E(v) = mc^{2} \textrm{ y } E(0) = m_{0}c^{2} \]

Esto lleva a que la masa relativista es una medida de la energía contenida en una partícula.

\[ E^{2} = c^{2}p^{2} + m_{0}^{2}c^{4} \textrm{ (en función de p)} \]

La energía intrínseca no altera la mecánica clásica porque se agregaría como suma en ambos lados de la ecuación.

\end{document}